\documentclass[9pt,a4paper,oneside, onecolumn]{article}
\usepackage[utf8]{inputenc}
\usepackage[english, german]{babel}
\usepackage{amsmath}
\usepackage{amsfonts}
\usepackage{amssymb}
\usepackage{graphicx}
\usepackage[left=2cm,right=2cm,top=2cm,bottom=2cm]{geometry}
\usepackage{gchords}
\usepackage{multicol}
\usepackage{mdframed} 
\author{Monsters Of Liedermaching}
\title{Der Fisch}
\pagestyle{empty}
\date{}

\begin{document}

\maketitle
\thispagestyle{empty}

\mbox{
\chord{t}{x,p3,p2,n,p1,n}{C}
\chord{t}{n,p2,p2,n,n,n}{Em}
\chord{t}{p3,p2,n,n,n,p3}{G}
\chord{t}{x,x,n,p2,p3,p2}{D}
}\\

\newcommand{\C}{\upchord{C}}
\newcommand{\Em}{\upchord{Em}}
\newcommand{\G}{\upchord{G}}
\newcommand{\D}{\upchord{D}}
\newcommand{\x}[1]{\upchord{#1}}

\raggedcolumns
\begin{multicols}{2}


\begin{verse}
Der \G{}Fisch war krank, der \Em{}Fisch war krank\\
Jetzt \C{}schwimmt er wieder, Gott sei Dank\\
Und \G{}dennoch geht's ihm \Em{}immer noch \C{}beschissen\\
Ein \G{}Karpfen-Girl aus \Em{}Übersee aus \C{}seiner letzten Fisch-WG\\
Hat \G{}tiefe Narben \Em{}in sein Herz \C{}gerissen\\
\end{verse}

\begin{verse}
Der \D{}Fisch war krank, der \G{}Fisch war krank\\
Er \C{}sitzt auf der Korallenbank\\
Und \D{}denkt an diesen \G{}schönen Tag im \C{}Mai\\
Sie \D{}war glitschig, \G{}sie war grün\\
Mit \C{}anderen Worten wunderschön\\
Und \G{}so schwamm sie im \Em{}Mai an ihm \C{}vorbei\\
\end{verse}

\begin{verse}
Er \G{}sagte nur: „Ey, \Em{}süße Maus, du \C{}siehst ja echt zum Blubbern aus!“\\
Und er schenkt ihr einen feuchten Blick\\
Sie verstand ihn wunderbar\\
Auch wenn sie nicht aus Ostsee war\\
Und am Abend gab's den ersten Fick\\
\end{verse}

\begin{verse}
Das \D{}geht bei Fischen \G{}ziemlich schnell\\
Auch \C{}wenn man sich eventuell\\
Dabei mal in den falschen Fisch verknallt\\
Doch diesmal schien es gut zu gehen\\
Sie wollte ihn danach noch sehen\\
Und ab \G{}da kannte die \Em{}Liebe keinen \C{}Halt\\
\end{verse}

\begin{verse}
\G{}Und sie schwammen \Em{}durch das Meer\\
\C{}Rauf und runter, kreuz und quer\\
Und sie blubberten die schönsten Blasen\\
Sie wohnten unten an der Klippe\\
Und rieben sich die Oberlippe\\
Fische haben keine Nasen\\
\end{verse}

\begin{verse}
Doch \D{}irgendwas schien \G{}nicht ok\\
\C{}Mit dem Girl aus Übersee\\
Das merkte man ganz deutlich nach und nach\\
Sie kam nachts nicht heim und log ihn an\\
Und roch nach fremdem Lebertran\\
Das \G{}war's an dem sein \Em{}kleines Glück \C{}zerbrach\\
\end{verse}

\begin{verse}
Und von \G{}diesem Tag an \Em{}ging's ihm schlecht\\
Ein \C{}japanischer Sumo-Hecht,\\
der wollte seine Karpfen-Lady küssen\\
Und besonders scheiße ist es dann, wenn man kein Karate kann\\
Tja, das hätte er eigentlich wissen müssen\\
\end{verse}

\begin{verse}
Die \D{}Gräten hin, das \G{}Glück passé\\
Das \C{}Karpfen-Girl aus Übersee war weg\\
Und sie ist es auch geblieben\\
Er hat's gehört, sie hat geschrieben\\
Sie ist bei diesem Hecht geblieben\\
Und \G{}wohnt jetzt \Em{}bei den \C{}Malediven\\
\end{verse}

\begin{verse}
Der \G{}Fisch war krank, der \Em{}Fisch war krank\\
Der \C{}Fisch war krank, der Fisch war krank\\
Jetzt schwimmt er wieder, Gott sei Dank\\
Der Fisch war krank, der Fisch war krank\\
Der Fisch war krank, der Fisch war krank\\
Jetzt schwimmt er wieder, Gott sei Dank\\

Gott sei \G{}Dank\\
\end{verse}
\end{multicols}
\end{document}
