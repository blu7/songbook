\documentclass[9pt,a4paper,oneside, onecolumn]{article}
\usepackage[utf8]{inputenc}
\usepackage[english, german]{babel}
\usepackage{amsmath}
\usepackage{amsfonts}
\usepackage{amssymb}
\usepackage{graphicx}
\usepackage[left=2cm,right=2cm,top=2cm,bottom=2cm]{geometry}
\usepackage{gchords}
\usepackage{multicol}
\usepackage{mdframed} 
\author{Funny van Dannen}
\title{Was ist mit der Liebe}
\pagestyle{empty}
\date{}

\begin{document}

\maketitle
\thispagestyle{empty}

\mbox{
\chord{t}{x,x,n,p2,p3,p2}{D}
\chord{2}{f1p1,p3,p3,p2,f1p1,f1p1}{F$\sharp$m}
\chord{t}{n,p2,p2,n,n,n}{Em}
\chord{t}{x,n,p2,p2,p2,n}{A}
}

\newcommand{\Am}{\upchord{Am}}
\newcommand{\A}{\upchord{A}}
\newcommand{\Dm}{\upchord{Dm}}
\newcommand{\Ds}{\upchord{D7}}
\newcommand{\F}{\upchord{F}}
\newcommand{\Fism}{\upchord{F$\sharp$m}}
\newcommand{\Fisdims}{\upchord{F$\sharp$7dim}}
\newcommand{\Hm}{\upchord{Hm}}
\newcommand{\Hbs}{\upchord{H$\flat$7}}
\newcommand{\C}{\upchord{C}}
\newcommand{\Cism}{\upchord{C$\sharp$m}}
\newcommand{\Em}{\upchord{Em}}
\newcommand{\E}{\upchord{E}}
\newcommand{\Es}{\upchord{E7}}
\newcommand{\Gs}{\upchord{G7}}
\newcommand{\G}{\upchord{G}}
\newcommand{\D}{\upchord{D}}
\newcommand{\Dms}{\upchord{Dm7}}
\newcommand{\x}[1]{\upchord{#1}}
\newcommand{\wdh}[1]{$\vert{}:$\ #1\ $:\vert$}

\begin{document}

\begin{verse}
Du tust so als wärst du ein Fremder, dabei kennst du dich ganz genau aus
Du tust so als wärst du ein Fremder dabei bist du hier zu Haus
\end{verse}
\begin{verse}Du gehst wie ein Gast durch die Räume als hättest du sie nie gesehn
Du hörst meine fragenden Worte als könntest du sie nicht verstehn
\end{verse}
\begin{verse}Du tust so als wärst du ein Fremder und du ziehst deinen Mantel nicht aus
Du tust so als wärst du ein Fremder dabei bist du hier zu Haus
\end{verse}
\begin{verse}Dann schaust du mich an und dann hälst du deinen Kopf wie ein tragischer Held
Du tust so als wärst du geheimnisvoll und der traurigste Mensch auf der Welt
\end{verse}
\begin{verse}Du tust so als wärst du ein Fremder und du ziehst deinen Mantel nicht aus
Du tust so als wärst du ein Fremder dabei bist du hier zu Haus
\end{verse}

\begin{verse}Du kratzt dich an deinem Dreitagebart warum willst du denn schon wieder gehen
Das Leben ist kein Video wer sollte es denn drehn
\end{verse}
\begin{verse}Du tust so als wärst du ein Fremder und ich tu so als wär ich allein
Warum tun wir beide nicht einfach so als könnten wir glücklich sein
\end{verse}
\begin{verse}
Du tust so als wärst du ein Fremder und du ziehst deinen Mantel nicht aus
Du tust so als wärst du ein Fremder dabei bist du hier zu Haus
\end{verse}
\end{document}
